\documentclass[eso]{bcc}

\newcommand{\citeabnt}[1]{(\textsc{\citeauthor{#1}}, \citeyear{#1})}
\newcommand{\citetabnt}[1]{\citeauthor{#1} (\citeyear{#1})}

\titulo{RELATÓRIO DE ESTÁGIO SUPERVISIONADO OBRIGATÓRIO (ESO)}

\palavrasChave{Automação}{Agilidade}{Auto atendimento}
\keywords{keyword1, keyword2, keyword3.}

\autor{Armstrong Lohãns de Melo Gomes Quintino}{lohansdemelo1108@gmail.com}

\orientador{Nome Professor}{}{UFAPE}
\orientadorDois{Nome Professor}{}{UFAPE}


\examinador{John Hopcroft}{}{UFAPE}
\examinadorDois{Richard Karp}{}{UFAPE}
\examinadorTres{Stephen Cook}{}{UFAPE}

\dataMesAno{31}{maio}{2022}

\begin{document}

\selectlanguage{portuguese}

\empresaNome{Google}
\empresaArea{Web}
\empresaPeriodo{7 de janeiro a 7 de julho de 2016}
\empresaCargaH{40H}
\empresaRemuneracao{R\$ 1.000,00}
\empresaSupervisorEmail{tanembaum@googlgle.com}
\logo{Figuras/UFAPE_logo.png}{0.88}

\capa % With error if not completely filled

% \capaDois

\begin{resumo}
Descrição do seu trabalho
\end{resumo}

\selectlanguage{english}
\begin{abstract}
Works Description
\end{abstract}
\selectlanguage{portuguese}

% Centralizar titulos
\renewcommand\contentsname{\centerline{Sumário}}
\renewcommand\listfigurename{\centerline{Lista de Figuras}}
\renewcommand\listtablename{\centerline{Lista de Tabelas}}

\lhead{Sumário}
\tableofcontents

\listoffigures
\addcontentsline{toc}{chapter}{Lista de Figuras}

\listoftables
\addcontentsline{toc}{chapter}{Lista de Tabelas}

\inicio
\chapter{Introdução}

Texto

\section {Objetivo}

O objetivo geral


\section{Justificativa}

Este trabalho se justifica.

\section{Organização do TCC}


\chapter{Fundamentação teórica}
\label{chap:fundamentacao}

Nesta

\chapter{A proposta do trabalho}
\label{chap:sistema}
A proposta deste trabalho

\chapter{Conclusões}
\label{chap:conclusao}

Considerações Finais


\nocite{*} % Ignore citations (remove if have any)
\bibliographystyle{estilo_ABNT}
\bibliography{refs}
\addcontentsline{toc}{chapter}{Referências Bibliográficas}

\appendix
\chapter{Apêndice}\label{ane:relatorio}
Este é o Apêndice.

\end{document}

