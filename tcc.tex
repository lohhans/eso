\documentclass[eso]{bcc}

\newcommand{\citeabnt}[1]{(\textsc{\citeauthor{#1}}, \citeyear{#1})}
\newcommand{\citetabnt}[1]{\citeauthor{#1} (\citeyear{#1})}

\titulo{RELATÓRIO DE ESTÁGIO SUPERVISIONADO OBRIGATÓRIO (ESO)}

\palavrasChave{Internet das Coisas}{Automação}{Agronegócio 4.0.}
\keywords{Internet of Things, Automation, Agrobusiness 4.0.}

\autor{Armstrong Lohãns de Melo Gomes Quintino}{lohansdemelo1108@gmail.com}

\orientador{Rodrigo Rocha}{Engenharia de Software}{UFAPE}
% \orientadorDois{Nome Professor}{Para completar}{UFAPE}


\examinador{John Hopcroft}{Para completar}{UFAPE}
\examinadorDois{Richard Karp}{Para completar}{UFAPE}
\examinadorTres{Stephen Cook}{Para completar}{UFAPE}

\dataMesAno{29}{agosto}{2022}

\begin{document}

\selectlanguage{portuguese}

\empresaNome{Google}
\empresaArea{Web}
\empresaPeriodo{7 de janeiro a 7 de julho de 2016}
\empresaCargaH{40H}
\empresaRemuneracao{R\$ 1.000,00}
\empresaSupervisorEmail{tanembaum@googlgle.com}
\logo{Figuras/UFAPE_logo.png}{0.88}

\capa % With error if not completely filled

% \capaDois

\begin{resumo}
Na atualidade está ocorrendo um aumento na utilização de diversas tecnologias e que, 
pelo aumento do seu uso, tem gerado um grande volume de informações. Esse aumento de 
informação foi devido a utilização da internet e, principalmente, pela utilização de 
diversos dispositivos que se comunicam. Por causa disso e pela inserção dessas tecnologias 
o termo internet das coisas se tornou conhecido em diversas áreas. Desta forma, a proposta 
deste projeto de pesquisa é ressaltar  a  importância  da Internet das Coisas e seus 
impactos econômicos e sociais, por meio de uma revisão bibliográfica e estudos acerca 
da Internet das Coisas aplicada no agronegócio bem como o desenvolvimento de uma aplicação 
simulando uma situação real. E para atingir a finalidade desses impactos diversas bases 
foram utilizadas a fim de embasar o conhecimento sobre a relação entre o paradigma da 
IoT aplicada ao agronegócio, sendo utilizada a pesquisa bibliográfica para este fim. 
Deste modo, o presente trabalho propõe-se a desenvolver uma aplicação simulando uma 
situação real além de compreender o impacto do uso da Internet das Coisas no agronegócio 
com o propósito de auxiliar aos gestores, por meio de informações precisas, o seu processo 
de tomada de decisão.
\end{resumo}

\selectlanguage{english}
\begin{abstract}
Currently, there is an increase in the use of various technologies and, due to 
the increase in their use, they have generated a large volume of information. 
This increase in information was due to the use of the internet and, mainly, 
the use of various devices that communicate. Because of this and the insertion 
of these technologies, the term internet of things has become known in several areas. 
Thus, the purpose of this research project is to emphasize the importance of the 
Internet of Things and its economic and social impacts, through a literature review 
and studies on the Internet of Things applied in agribusiness as well as the development 
of an application simulating a situation real. And to achieve the purpose of these impacts, 
several bases were used in order to base knowledge about the relationship between 
the IoT paradigm applied to agribusiness, using bibliographic research for this purpose. 
In this way, the present work proposes to develop an application simulating a real 
situation in addition to understanding the impact of the use of the Internet of Things 
in agribusiness in order to help managers, through accurate information, your decision-making process.
\end{abstract}
\selectlanguage{portuguese}

% Centralizar titulos
\renewcommand\contentsname{\centerline{Sumário}}
\renewcommand\listfigurename{\centerline{Lista de Figuras}}
\renewcommand\listtablename{\centerline{Lista de Tabelas}}

\lhead{Sumário}
\tableofcontents

\listoffigures
\addcontentsline{toc}{chapter}{Lista de Figuras}

\listoftables
\addcontentsline{toc}{chapter}{Lista de Tabelas}

\inicio
\chapter{Introdução}

O Brasil, na atualidade, é um dos maiores produtores de alimento do mundo, com potencial para 
ser o maior produtor mundial pelo fato de dispor de vários recursos, principalmente climáticos, 
que favorecem a vasta produção de alimentos.

Além do fator climático, o Brasil apresenta quantidade de água considerável e potencial de 
mais áreas agricultáveis, pois atualmente se utiliza apenas 7,3\% dessas áreas.Associado a isso, 
há mais investimentos em tecnologia, o que difere positivamente nos valores de produção alcançados.

O incremento na utilização de tecnologias se deve ao fato da inserção do termo indústria 4.0, 
surgida na Alemanha, onde, resumidamente, considera-se que a tecnologia digital aplica-se em todos 
os aspectos da sociedade. Desta forma, a agricultura seguiu o termo e se intitula agricultura 4.0, 
pois a mesma se beneficia dos avanços tecnológicos a fim de suprir e melhorar suas necessidades 
de produção na busca de encontrar novas formas de tornar o negócio mais eficiente e competitivo.

Assim, essa nova era agricultura 4.0 trouxe consigo novos questionamentos e, principalmente, 
houve o aumento da preocupação em utilizar o mínimo possível dos recursos naturais e continuar 
produzindo cada vez um volume maior que uma região possa oferecer. Logo, a indústria 4.0 surge 
como uma aliada para otimizar esse sistema e proporcionar essas melhorias.

Diante desse cenário, verifica-se que o Brasil ocupa a posição de 2º maior produtor de alimentos 
do planeta, depois dos Estados Unidos. Uma pesquisa feita em 2017 pela Comissão Brasileira de 
Agricultura de Precisão (CBAP), vinculada ao Ministério da Agricultura, constatou que 67\% das 
propriedades agrícolas do país já utilizam algum tipo de tecnologia na área de gestão do negócio 
e nas atividades de cultivo e colheita da produção. Desta forma, a produção agroindustrial é 
e continua sendo uma válvula de escape fundamental contra a crise econômica que atingiu o Brasil 
nos últimos anos. Em 2015, o setor empregava 19 milhões de pessoas. No ano seguinte, houve aumento 
em cerca de 75 mil novos empregos, segundo dados da Confederação da Agricultura e Pecuária do Brasil 
(CNA) e do Centro de Estudos Avançados em Economia Aplicada (CEPEA).

Ou seja, para possibilitar o agronegócio 4.0 foi necessário conectar coisas com a internet que 
foi chamado de Internet das Coisas (IoT). Desde então, a Internet das Coisas vem se difundindo 
nos mais diferentes ambientes, do meio empresarial ao cotidiano do homem comum estando presente 
nos mais diversos dispositivos que possuem conexão com a internet. Um ponto importante, 
destacado na literatura, é que a IoT tem o potencial de criar um impacto econômico de 
US\$ 2,7 trilhões para US\$ 6,2 trilhões até 2025. Alguns dos usos mais promissores são os 
cuidados com a saúde, as infraestruturas e os serviços do setor público, ajudando a sociedade 
a enfrentar alguns dos seus maiores desafios. (MANYIKA et al., 2013, p. 51).

A partir destes fatos, o objetivo geral deste projeto de pesquisa é apresentar um panorama 
sobre novas possibilidades da IoT dentro do setor do agronegócio e, em específico, 
contextualizar a IoT de forma a levantar, relacionar e fazer uma análise preliminar dessas 
novas possibilidades.Além disso, ressaltar  a  importância  da Internet das Coisas e seus 
impactos econômicos e sociais,  por meio de uma revisão bibliográfica e estudos acerca da 
Internet das Coisas aplicada no agronegócio bem como o desenvolvimento de uma aplicação 
simulando uma situação real.


\section {Objetivo}

\begin{enumerate}
    \item Geral\\
    O objetivo deste projeto é analisar as novas possibilidades geradas pela Internet das Coisas para o agronegócio
    \item Específico
    \begin{enumerate}
        \item[$-$]  Entender a estrutura da internet das coisas
        \item[$-$] Entender o agronegócio
        \item[$-$] Levantar as novas possibilidades com o uso da Internet das Coisas
        \item[$-$] Relacionar as novas possibilidades pela IoT para o agronegócio 
        \item[$-$] Fazer uma análise preliminar dessas novas possibilidades
        \item[$-$] Desenvolver um aplicativo simulando uma situação real
    \end{enumerate}
\end{enumerate}


\section{Justificativa}

Este trabalho se justifica.

\section{Organização do TCC}


\chapter{Fundamentação teórica}
\label{chap:fundamentacao}

Nesta

\chapter{A proposta do trabalho}
\label{chap:sistema}
A proposta deste trabalho

\chapter{Conclusões}
\label{chap:conclusao}

Considerações Finais


\nocite{*} % Ignore citations (remove if have any)
\bibliographystyle{estilo_ABNT}
\bibliography{refs}
\addcontentsline{toc}{chapter}{Referências Bibliográficas}

\appendix
\chapter{Apêndice}\label{ane:relatorio}
Este é o Apêndice.

\end{document}

